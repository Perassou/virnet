% Chapter 3

\chapter{Deep neural networks for identification} % Main chapter title

\label{Chapter3} % For referencing the chapter elsewhere, use \ref{Chapter1} 

\lhead{Chapter 3. \emph{Deep neural networks for identification}} % This is for the header on each page - perhaps a shortened title

%----------------------------------------------------------------------------------------


\section{Convolution neural networks}

\LaTeX{} is not a WYSIWYG (What You See is What You Get) program, unlike word processors such as Microsoft Word or Apple's Pages. Instead, a document written for \LaTeX{} is actually a simple, plain text file that contains \emph{no formatting}. You tell \LaTeX{} how you want the formatting in the finished document by writing in simple commands amongst the text, for example, if I want to use \textit{italic text for emphasis}, I write the `$\backslash$\texttt{textit}\{\}' command and put the text I want in italics in between the curly braces. This means that \LaTeX{} is a ``mark-up'' language, very much like HTML.

\subsection{Triplet loss mechanism}

\section{Sequence neural networks}

\subsection{Attention mechanism}
%----------------------------------------------------------------------------------------

\begin{flushright}
Guide written by ---\\
Sunil Patel: \href{http://www.sunilpatel.co.uk}{www.sunilpatel.co.uk}
\end{flushright}
